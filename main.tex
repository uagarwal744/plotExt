\documentclass[a4paper,10pt]{scrreprt}

\usepackage[left=10mm,right=10mm,top=1in,bottom=1in]{geometry}
\usepackage{pseudocode}
\usepackage{xcolor}
\usepackage{hyperref}
\usepackage{bookmark}
\usepackage{graphicx}
\usepackage{underscore}
\usepackage{amssymb}
\usepackage{pseudocode}
\usepackage{array}
\usepackage[utf8]{inputenc}
\setlength{\arrayrulewidth}{0.5mm}
\date{}


\begin{document} 

\begin{titlepage}
    \vspace*{\fill}
    \begin{center}
      \textcolor{black}{{\Huge {\textbf{Plot to Tables}}}}\\[0.5cm]
      \textcolor{black}{{\Huge {\textbf{Report}}}}\\[0.5cm]
    \end{center}
    \vspace*{\fill}
\end{titlepage}


\tableofcontents 

\chapter{Installation } 

\section{Downloading}

\paragraph{}
This document describes the software requirements, algorithm, results obtained on different test cases and future scope of AutoMosaic Software. This software is a standalone application designed to restore a torn document from the pieces. The specified requirements are of high priority and AutoMosaic Software works best when they are followed.

\section{Installing Prerequisites} 
\paragraph{}


\section{Installing ....} 




\chapter{User Manual} 

\section{Running the Application}
\paragraph{}
This software takes individual images of the torn fragments of a document as input, and returns an output merged image. Mosaicing of torn documents is an important and indispensable labour in the field of the forensics or archival study. The manual process being quite tedious, automation of the same utilizing this software can result in a phenomenal increase in efficiency.



\section{User Interface}
\paragraph{}
This software has been exported into an executable file and hence is compatible with Windows OS. It runs on all Windows OS version above Windows XP.

\section{Keyboard Shortcuts}

\begin{itemize}

\item The images of the fragments, which are provided as the input must be taken against a white background for an efficient extraction of the fragment contours.

\item There is a considerable amount of error in presence of white regions inside the fragment contour.

\item To tackle the issue of memory overload due to high resolution images, we resize the images for convenience and final image is constructed retaining original image size, which may result a small amount of inaccuracy.

\end{itemize}
  

\section{Functions And their usage}
\paragraph{}
A user manual in pdf format would be made available along with the software.
\begin{itemize}
   \item Document pieces do not contain any straight line cut as they are hand cut.
   \item Documents are not torn using a shredder.
\end{itemize}

\chapter{Description of Algorithms used}

\section{Getting all images from the pdf}
\subsection{Converting pdf file to images}
\paragraph{}
To extract all the graphs from a pdf file, first convert all the pdf pages to images. To do this two modules have been used namely PythonMagick and pyPdf. And PdfFileReader is imported from pyPdf module.
\paragraph{}
Firstly a blank image file has been created using PythonMagick and required density is assigned to it. When the pdf page is read using this image file, it reads in the specified density. After that all pdf pages are written in image format.
\\
\begin{pseudocode}{\textbf{pdf_to_img}}{ }
/\mbox{*}$ Creating an image file using PythonMagick and assigning density to it.$\\ $img.read reads the input\_file in the specified density(say 200) and $\\$img.write writes the pdf file in image format. $ \mbox{*}/\\
$img $ \GETS $ PythonMagick.Image() $\\
$img.density(density) $\\
$my\_file $ \GETS $ PdfFileReader(file("\{0\}".format(input\_file),"r")) $\\
$numpages $ \GETS $ my\_file.getNumPages() $\\
$file $ \GETS $ [ ] $\\ 
\FOR i \GETS 0 \TO numpages \DO
\BEGIN
$img.read("\{0\}[\{1\}]".format(input\_file,i))$\\
$img.wrire("\{0\}\_image\{1\}.jpg".format(input\_file,i))$\\
$file.append("\{0\}\_image\{1\}.jpg".format(input\_file,i))$\\
\END\\
\end{pseudocode}
\subsection{Extracting graphs from an image}
\paragraph{}
After converting the pdf file to images all the graphs are extracted from each image file. To do this six modules have been used namely cv2, numpy, math, wand.image, os, sys. Image has been imported from wand.image.
\paragraph{}
At first a function is defined which does the probabilistic Hough transform on the image given. In this the image file is read and converted to a gray-scale image using cv2.cvtColor(). After that Adaptive Thresholding, which converts gray-scale image to binary one, is done using cv2.adaptiveThreshold() followed by morphological transformation using cv2.morphologyEx(). And then Image blurring/smoothing is done using cv2.GaussianBlur() followed by straight lines detection using cv2.HoughLinesP().
\paragraph{}
After the Hough Transform is done, a function is defined to approximate the shape on the contours detected. In this function contours are detected using cv2.findContours and stored in a list. Now for each contour detected in this list contour area is calculated using cv2.contourArea() and is compared with the threshold area. If this area is greater than the threshold area, the shape of the contour is approximated using cv2.approxPolyDp(). /* If the length of this approximated shape is between two and ten the coordinates of the bounding rectangle of this contour is found using cv2.boundingRect().*/ After doing this the coordinates of the graph were found and this portion of the image is stored in the graphArray and all such graphs were appended to this array.
\begin{pseudocode}{\textbf{pdfwithgraph}}{ }
$def \textbf{houghp} $\\
/\mbox{*}$Does a Hough Transform (probabilistic) on the image given$\mbox{*}/\\
$img$\GETS$cv2.imread(self.pageImage)$\\
$gray$\GETS$cv2.cvtColor(img,cv2.COLOR\_BGR2GRAY$)\\
$h,w,channel$\GETS$img.shape$\\
$gray$\GETS$cv2.adaptiveThreshold(gray,255,cv2.ADAPTIVE\_THRESH\_MEAN\_C,cv2.THRESH\_BINARY\_INV,11,7)$\\
$kernel$\GETS$ np.zeros((5,5),np.uint8)$\\
$gray$\GETS$cv2.morphologyEx(gray, cv2.MORPH\_OPEN, kernel)$\\
$gray$\GETS$cv2.GaussianBlur(gray,(5,5),0)$\\
$lines$\GETS$cv2.HoughLinesP(gray,self.det_prec,np.pi/self.ang_prec,100,self.minLineLength,self.maxLineGap)$\\
$im2$\GETS$np.zeros((h,w,channel),np.uint8)$\\
\FOREACH x1,y1,x2,y2\in lines[0] \DO
\BEGIN 
$cv2.line(im2,(x1,y1),(x2,y2),(0,255,0),2)$\\
$\textbf{return} im2$\\
\END \\
\\
$def \textbf{polydp}(image)$\\
/\mbox{*}$Does a polydp (shape) approximation on the contours detected$\\
$image--image returned by the houghp()$\mbox{*}/\\
$img$\GETS$cv2.imread(self.pageImage)$\\
$gray$\GETS$cv2.cvtColor(image,cv2.COLOR_BGR2GRAY)$\\
$(cnts, _)$\GETS$cv2.findContours(gray, cv2.RETR_EXTERNAL,cv2.CHAIN_APPROX_SIMPLE)$\\
$h,w,channel$\GETS$image.shape$\\
$hi,wi$\GETS$h,w$\\
$count_graph$\GETS$1$\\
\FOREACH c \in cnts\DO
\BEGIN 
\IF cv2.contourArea(c) >self.threshArea \THEN
\BEGIN 
$approx$\GETS$cv2.approxPolyDP(c,0.01*cv2.arcLength(c,True),True)$\\
\IF len(approx)<10 and len(approx)>2 \THEN
\BEGIN
$x,y,w,h$\GETS$cv2.boundingRect(c)$\\
\IF aspectRatio(w,h) is equal to True and h < 0.95*hi and w < 0.95*wi \THEN
\BEGIN 
$percentin$\GETS$0.2$\\
$xstart$\GETS$max(0,x-percentin*w)$\\
$xend$\GETS$min(wi,x+w+percentin*w)$\\
$ystart$\GETS$max(0,y-percentin*h)$\\
$yend$\GETS$min(hi,y+h+percentin*h)$\\
$graph$\GETS$img[int(ystart):int(yend),int(xstart):int(xend)]$\\
$graphArray.append(graph)$\\
$graphName.append(self.graphfolder+"/"+str(count\_graph)+".png")$\\
$cv2.imwrite(self.graphfolder+"/"+str(count\_graph)+".png",graph)$\\
$count_graph$\GETS$count_graph + 1$\\
\END 
\END
\END
\END\\
\\
$def \textbf{aspectRatio}(w,h)$\\
\IF h*1.0/w\ge4 or w*1.0/h\ge4 \THEN
\BEGIN
$\textbf{return} False$\\
\END \\
$\textbf{return} True$\\
\\
$def \textbf{graphextract}$\\
\FOR countpage\GETS 0 \TO numPages\DO
\BEGIN
$graphfolder$\GETS$"output\_page\_"+str(countPage)$\\
\IF \textbf{not} os.path.exists(graphfolder) \THEN
\BEGIN 
$os.makedirs(graphfolder)$\\
$pageImage$\GETS$pdfImage[countPage]$\\
$polydp(houghp())$\\
\END 
$\textbf{return} graphArray,graphName$\\
\END 
\end{pseudocode}


\section{Getting different color plot projections from an image}
\subsection{Python 2.7}
\paragraph{}
Python is a widely used general-purpose, high-level programming language.
Python supports multiple programming paradigms, including object-oriented, imperative and functional programming or procedural styles. It features a dynamic type system and automatic memory management and has a large and comprehensive standard library.
\paragraph{}
Like other dynamic languages, Python is often used as a scripting language, but is also used in a wide range of non-scripting contexts. Using third-party tools, such as Py2exe, or Pyinstaller, Python code can be packaged into standalone executable programs. Python interpreters are available for many operating systems.

\section{Getting the scale from the graph}
\paragraph{}
For the scale detection part at first the axis of the graph was determined and then its coordinates were used to extract out the scales. This determination of the coordinates of the axis was done via the hough transfrom algorithm.The corresponding function used was cv2.HoughLinesP.
Now once the axis was determined the image containing the graph was cropped below the coordinates of the lower x – axis and then the OCR was done on this cropped image to get out the x-scale values.
\paragraph{}
For the extraction of the y-scale,as done previously the original image was cropped such that the new image contained only the region left to the leftmost axis.Now the OCR could not be done directly on this cropped image as it contains texts to the left of the scale.So to crop out the texts the image was first rotated by 90 degrees in the clockwise direction.The corresponding functions used for rotation were cv2.transpose
 and cv2.flip. Now on this rotated image OCR was run to determine the coordinates of the text.This text region was then cropped out of the image and then this image was re-roted again by 90 degrees in the clockwise direction.OCR was finally done on this image to return the corresponding y-scale values of the graph.
\section{Getting the table values}
\paragraph{}
To obtain tables from plots, it is assumed that the plots are binary images. For this, a K means clustering algorithm is applied to the graphs and binary colour detected images are found out. Each of these images are expected to contain the plot of a single variable. Each of these plots are sent as input to the function approxTable(). The function approxTable() returns the table from the plot of a single variable. The function takes the binary image of the plot as input. We traverse the image along the x-axis, and for a particular x, the value for y=f(x) is found by traversing along the y-axis and finding out the white pixel in that particular column. If there are multiple white pixels for a given x, the value of y=f(x) is the average of those pixel values. If there are no white pixels detected for a particular x, then the function has a discontinuity at that point and a "null" value is returned.
\paragraph{}
To remove the discontinuities created due to noise or if the graph lines are dotted, line interpolation is used. The line interpolation function joins discontinuities by a straight lines if the length of the discontinuity is below a particular threshold distance. This is done to overcome the problems created if the plot lines are dotted or due to noise. The  function "lineInterpolation_x" and "lineInterpolation_y" serve this purpose.
\paragraph{}
The function approxtable() is used to get the tabulated form of the plot. To this function interpolated plots are given as inputs. The parameters are interpolated plot image, number of pixels between two consecutive x-axis divisions, number of pixels between two consecutive y-axis, the length of the bounding box of the graph along the x-axis, the length of the bounding box of the graph along y-axis, the coordinate from where x-axis starts, the coordinate from where y-axis starts, units per division on the x-axis, units per division on the y-axis.
The function starts from the leftmost point on the x-axis and each time proceeds one-tenth of the minimum marked division on the x-axis, and finds the corresponding y=f(x) value in pixels. Then from the scale values the pixels are converted into actual values. The function then returns two lists - x and f(x).
  
\chapter{Code Modules and Architectural Diagram}

\chapter{Test Plan}
\section{Introduction}
\paragraph{}
Its objective is to communicate project-wide quality standards and procedures. It portrays a snapshot of the project as of the end of the planning phase. This document will address the different standards that will apply to the unit, integration and system testing of the specified application. Testing criteria under the white box, black box, and system-testing paradigm will be utilized. This paradigm will include, but is not limited to, the testing criteria, methods, and test cases of the overall design.
\section{Test Objective}
\paragraph{}
The objective of this test plan is to find and report as many bugs as possible to improve the integrity of our program. Although exhaustive testing is not possible, a broad range of tests will be exercised to achieve the goal. There will be following functions that can be performed by this application: Search Author, Search Journal, Refine results, view and manage favorites, view recent history. The user interface to utilize these functions is designed to be user-friendly and provide easy access to all the functions.
\section{Process Overview}
\paragraph{}
The following represents the overall flow of the testing process:
\begin{enumerate}
    \item Identify the requirements to be tested. All test cases shall be derived using the current Program Specification.
    \item Identify which particular test(s) will be used to test each module.
    \item Review the test data and test cases to ensure that the unit has been thoroughly verified and that the test data and test cases are adequate to verify proper operation of the unit.
    \item Identify the expected results for each test.
    \item Document the test case configuration, test data, and expected results.
    \item Perform the test(s).
    \item Document the test data, test cases, and test configuration used during the testing process. This information shall be submitted via the Unit/System Test Report (STR).
    \item Successful unit testing is required before the unit is eligible for component integration/system testing.
    \item Unsuccessful testing requires a Bug Report Form to be generated. This document shall describe the test case, the problem encountered, its possible cause, and the sequence of events that led to the problem. It shall be used as a basis for later technical analysis.
    \item Test documents and reports shall be submitted. Any specifications to be reviewed, revised, or updated shall be handled immediately.
\end{enumerate}

\chapter{Testing Process}
\paragraph{}
The diagram above outlines the Test Process approach that will be followed.
\begin{enumerate}
    \item Organize Project involves creating a System Test Plan, Schedule and Test Approach, and assigning responsibilities.
    \item Design/Build System Test involves identifying Test Cycles, Test Cases, Entrance and Exit Criteria, Expected Results, etc. Test Cases and the Data required are identified. The Test conditions are derived from the Software Requirement Specifications.
    \item Design/Build Test Procedures includes setting up procedures such as Error Management systems and Status reporting.
    \item Build Test Environment includes requesting/building hardware, software and data setups.
    \item Execute System Tests - The tests identified in the Design/Build Test Procedures will be executed. All results will be documented and Bug Report Forms filled out and given to the Development Team as necessary.
    \item Signoff - Signoff happens when all pre-defined exit criteria have been achieved.
\end{enumerate}
\chapter{Testing Strategy}
\paragraph{}
The following outlines the types of testing that were done for unit, integration, and system testing. It includes what will be tested, the specific use cases that determine how the testing is done.
\section{Unit Testing}
\paragraph{}
Unit Testing is done at the source or code level for language-specific programming errors such
as bad syntax, logic errors, or to test particular functions or code modules. The unit test cases
shall be designed to test the validity of the programs correctness.
\subsection{White Box Testing}
\paragraph{}
In white box testing, the UI is bypassed. Inputs and outputs are tested directly at the code level and the results are compared against specifications. This form of testing ignores the function of the program under test and will focus only on its code and the structure of that code. Test cases are generated that not only cause each condition to take on all possible values at least once, but that cause each such condition to be executed at least once.
\\
Yet to be written
\\
\subsection{Black Box Testing}
Black box testing typically involves running through every possible input to verify that it results in the right outputs using the software as an end-user would. We have decided to perform Error guessing and Boundary Value Analysis testing on our application.
\section{Integration Testing}
\subsection{Incremental Testing}
\paragraph{}
There are 'x' primary modules that were needed to be integrated. The module-1 module, the module-2 module etc etc. These components, once integrated form the complete Blah-Blah application. The following describes these modules as well as the steps that will need to be taken to achieve complete integration.We will be employing an incremental testing strategy to complete the integration.
\paragraph{}
\textbf{Module-1:}
\textbf{Module-2:}
\\
\\
\subsection{System Testing}
\paragraph{}
The goals of system testing are to detect faults that can only be exposed by testing the entire integrated system or some major part of it. Generally, system testing is mainly concerned with areas such as performance, security, validation, load/stress, and configuration sensitivity. But in our case we focused only on function validation and performance. And in both cases we used the black-box method of testing.
\subsection{Function Validation Testing}
The integrated Blah-Blah Application was tested based on the requirements to ensure that we built the right application. In doing this test, we tried to find the errors in the inputs and outputs, that is, we tested each function to ensure that it properly implements the parsing procedures, and that the results are expected. The behavior of each function are contained in the Software Requirement Specification.
\paragraph{}
In addition, we tested:
\begin{itemize}
	\item The interfaces to ensure they are functioning as desired (i.e. check if each interface is behaving as expected, specifically verifying the appropriate action is associated with each mouse_click event).
	\item Some more stuff.
\end{itemize}
\subsection{Performance Testing}
\paragraph{}
This test was conducted to evaluate the fulfillment of a system with specified performance requirements. It was done using black-box testing method. The following were tested:
\begin{itemize}
	\item Setting high number of graphs in the given pdf file and checking the time required for the exraction of all the graphs and tables for each plot.
	\item Some more stuff
\end{itemize}
\chapter{Entry and Exit Criteria}
\paragraph{}
This section describes the general criteria by which testing commences, temporarily stopped, resumed and completed within each testing phase. Different features/components may have slight variation of their criteria, in which case, those should be mentioned in the feature test
plan. The testing phase also maps to the impact level definition when a defect is entered in the bug-tracking phase.
\section{Unit Testing}
\paragraph{}
Unit Testing is done at the source or code level for language-specific programming errors such as bad syntax, logic errors, or to test particular functions or code modules. The unit test cases shall be designed to test the validity of the programs correctness.
\subsection{Black Box Phase}
\paragraph{}
Black box testing typically involves running through every possible input to verify that it results in the right outputs using the software as an end-user would. We will use Equivalence Partitioning and Boundary Value Analysis complexity metrics in order to quantifiably determine how
many test cases needed to achieve maximum code coverage.
\paragraph{}
\textbf{Black Box Entry Criteria}
\paragraph{}
The Black Box Entry Criteria will rely on the component specification, and user interface requirements. Things that must be done on entry to the Black Box stage:
\begin{itemize}
	\item All functions like so-and-so funtions must be either coded or stubs written.
	\item The type of Black Box testing Methods will be determined upon entry. We will use so-and-so analysis.
	\item Some more stuff
\end{itemize}
\paragraph{}
\textbf{Black Box Exit Criteria}
\paragraph{}
The Black Box Exit Criteria listed below explains what needs to be completed in-order to exit Black Box phase. To exit the Black Box phase 100\% success rate must be achieved. Things that must be done upon exiting the Black Box stage:
\begin{itemize}
	\item Some Stuff.
\end{itemize}
\paragraph{}
\subsection{White Box Testing}
\paragraph{}
The White Box criteria apply for purposes of focusing on internal program structure, and dis- cover all internal program errors. Defects will be categorized and the quality of the product will be assessed.
\\
\textbf{White Box Entry Criteria}
\paragraph{}
The White Box Entry Criteria will rely verifying that the major features work alone but not necessarily in combination; exception handling will not be implemented. The design and human interface are stable. Things that must be done on entry to the White Box stage:
\begin{itemize}
	\item All functions so-and-so must be coded.
	\item The type of White Box testing Methods was determined upon entry. We will use unit testing and test for memory leaks.
	\item Black Box Testing should be in its late stages.
	\item Some more stuff
\end{itemize}
\paragraph{}
\textbf{White Box Exit Criteria}
\paragraph{}
The Citation Analyzer in the White Box stage should have a generally stable feel to it. White Box testing continued until the Black Box or next milestone criteria were met. To exit the White Box phase 100\% success rate must be achieved. The following describes the state of the product upon exit from the White Box Stage:
\begin{itemize}
	\item All funtions like blaa-blah-blah are implemented and tested
	\item All Branch Testing test cases were generated. The test cases will be generated from the Control Flow diagrams of all functions.
	\item The graphical interface has been reviewed and found to satisfactory and is stable, that is, no further changes to dialog boxes or other interface elements are planned. Minor changes are acceptable, but must be arranged with the Development and Test Engineers.
	\item All code bugs that are exposed are corrected.
\end{itemize}
\section{Integration Testing}
\paragraph{}
There are X modules that will be integrated for Integration testing. Some more Stufffff.
\subsection{Integration Test Entry Criteria}
\paragraph{}
The Integration Test Entry Criteria will rely on these X modules to be operational. 
Some Stuffffff.
Things that must be done on entry to the Integration Test stage:
\begin{itemize}
	\item All funtions like blah-blah are either coded and/or stubs created.
	\item Some more stuff.
\end{itemize}
\paragraph{}
\subsecton{Integration Test Exit Criteria}
\paragraph{}
The Integration Test Exit Criteria relied on these X modules to be operational.To exit the Integration Testing phase 100\% success rate must be achieved. Things that must be done on exit from the Integration Test stage:
\begin{itemize}
	\item All code bugs that were exposed were corrected.
	\item Some Stuff.
	\item These X modules are ready for System Testing. Stubs and drivers are replaced with fully functional code.
	\item Black Box Testing was completed.
\end{itemize}
\section{System Testing}
\paragraph{}
The System Test criteria apply for purposes of categorizing defects and the assessing the quality level of the product. All elements of these so-and-so modules are meshed together and tested as a whole. System test focuses on functions and performance, reliability, installation, behavior during special conditions, and stress testing.
\subsection{System Test Entry Criteria}
\begin{itemize}
	\item Unit testing of all the modules has been completed.
	\item System test cases have been documented.
	\item UI and functionalities to be tested should be frozen.
	\item Priorities bugs have been fixed.
	\item Some more Stuff if needed.
\end{itemize}
\subsection{System Test Exit Criteria}
\begin{itemize}
	\item Application meets all the document requirements and functionalities.
	\item Defects found during System testing should be fixed and closed.
	\item All the test cases for the system have been executed and passed.
	\item No critical defects have been opened which is found during system testing.
	\item Some more stuff if needed.
\end{itemize}
\section{Shipping/Live Release}
\paragraph{}
Some Stuff.
\subsection{Shipping/Live Release Entry Criteria}
\paragraph{}
The criteria for entering the final stages are as follows:
\begin{itemize}
	\item Some Stuff
\end{itemize}
\subsection{Shipping/Live Release Exit Criteria}
\begin{itemize}
	\item No critical defects found.
	\item Some more Stuff.
\end{itemize}
\chapter{Deliverables}
\paragraph{}
Following is a list of deliverables for our software:
\begin{itemize}
	\item Program function specifications
	\item Program source code
	\item Test plan document - this document should address testing objectives, criteria, standards, schedule and assignments, and testing tools.
	\begin{itemize}
		\item Unit Testing Plan
		\item Integration Plan
		\item System Testing Plan
	\end{itemize}
\end{itemize}
\end{document}

